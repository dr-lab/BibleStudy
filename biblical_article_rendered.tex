\documentclass{article}
\usepackage[english]{babel}
\usepackage[letterpaper,top=2cm,bottom=2cm,left=3cm,right=3cm,marginparwidth=1.75cm]{geometry}
\usepackage{amsmath}
\usepackage{graphicx}
\usepackage[colorlinks=true, allcolors=blue]{hyperref}

\title{
Bible Study
}
\begin{document}
\maketitle


\section{Word List}

\begin{table}[h]
\centering
\begin{tabular}{|p{0.3\linewidth}|p{0.4\linewidth}|p{0.3\linewidth}|}
\hline
\textbf{Word} & \textbf{Meaning} & \textbf{Word Root and Meaning} \\
\hline
Relation & A connection between two or more people or things & Relate - to bring into or establish a relationship \\
\hline
Wife & A married woman & Wife - from Old English wif, meaning woman or wife \\
\hline
Husband & A married man & Husband - from Old Norse hús, meaning house, and bóndi, meaning dweller or farmer \\
\hline
Family & A group consisting of parents and children living together in a household & Familia - Latin for household servants \\
\hline
Respect & Admiration or esteem for someone or something & Respectus - Latin for regard, consideration \\
\hline
Convict & A person found guilty of a crime and sentenced by a court & Convictus - Latin for overcome, conquered \\
\hline
Convince & To cause someone to believe or do something & Convinco - Latin for prove or overcome \\
\hline
Control & To have power or authority over something or someone & Contrarotulus - Medieval Latin for register, list \\
\hline
Coerce & To compel or force someone to do something against their will & Coercere - Latin for restrain, control \\
\hline
Expectation & A strong belief that something will happen or be the case in the future & Expectatio - Latin for anticipation, hope \\
\hline
Love & An intense feeling of deep affection & Lufu - Old English for love, affection \\
\hline
\end{tabular}
\end{table}


\section{OpenAI Article}

\subsection{Biblical Article One}
The Power of Love in Family Relations

Love is essential in life as it creates an environment of trust, comfort, and security. In marriage, it is the foundation to a good \textbf{\textbf{relation}}ship between the \textbf{\textbf{husband}} and \textbf{\textbf{wife}}. The \textbf{\textbf{family}} unit is strengthened when \textbf{\textbf{love}} is shown between the members. Love is a fundamental virtue that should be nurtured, cherished, and esteemed.

A \textbf{\textbf{wife}} should \textbf{\textbf{respect}} her \textbf{\textbf{husband}} as the head of the \textbf{\textbf{family}}. The Bible says, "Wives, submit yourselves to your own \textbf{\textbf{husband}}s as you do to the Lord. For the \textbf{\textbf{husband}} is the head of the \textbf{\textbf{wife}} as Christ is the head of the church, his body, of which he is the Savior" (Ephesians 5:22-23, NIV). In submitting to her \textbf{\textbf{husband}}, she shows \textbf{\textbf{respect}} and trusts in his leadership. In \textbf{\textbf{respect}}ing her \textbf{\textbf{husband}}, the \textbf{\textbf{husband}} is \textbf{\textbf{respect}}ed in front of his children and society. 

The \textbf{\textbf{husband}} should \textbf{\textbf{love}} and honor his \textbf{\textbf{wife}} as he would his own body. "Husbands, \textbf{\textbf{love}} your wives, just as Christ \textbf{\textbf{love}}d the church and gave himself up for her to make her holy, cleansing her by the washing with water through the word" (Ephesians 5:25-26, NIV). His \textbf{\textbf{love}} for his \textbf{\textbf{wife}} should be unconditional, giving, and selfless. A \textbf{\textbf{wife}} who is \textbf{\textbf{love}}d like this would feel valued, cherished, and appreciated.

Family \textbf{\textbf{expectation}} and \textbf{\textbf{convict}}s should be communicated in \textbf{\textbf{love}}, and not through \textbf{\textbf{control}} or coercion. "Fathers, do not exasperate your children; instead, bring them up in the training and instruction of the Lord" (Ephesians 6:4, NIV). Parents should not push their children beyond their ability, or make unnecessary demands without showing affection. Parents and children alike should communicate with each other, be considerate, and listen to each other.

Love isn't just emotions or feelings, but also actions. In 1 Corinthians 13:4-5, NIV, it says, "Love is patient, \textbf{\textbf{love}} is kind. It does not envy, it does not boast, it is not proud. It does not dishonor others, it is not self-seeking, it is not easily angered, it keeps no record of wrongs." A \textbf{\textbf{family}} whose level of \textbf{\textbf{love}} is founded on this kind of \textbf{\textbf{love}} would not strive for power, \textbf{\textbf{control}}, or put undue pressure on each other.

In conclusion, Love creates an environment where there is mutual \textbf{\textbf{respect}}, healthy communication, and affection in the \textbf{\textbf{family}} unit. Love is shown in actions and not just words. Family members should strive to show each other \textbf{\textbf{love}} and build each other up. Following these biblical principles and applying them in \textbf{\textbf{family}} \textbf{\textbf{relation}}ships can lead to a happier and fulfilled life.

References
1. Ephesians 5:22-23, NIV
2. Ephesians 5:25-26, NIV
3. Ephesians 6:4, NIV
4. 1 Corinthians 13:4-5, NIV

\subsection{Biblical Article Two}
Title: Building Strong Relations in the Family through Love and Respect

As a biblical article writer, I want to share with you how to build strong \textbf{\textbf{relation}}s in the \textbf{\textbf{family}} through \textbf{\textbf{love}} and \textbf{\textbf{respect}}. One of the most critical aspects of a healthy \textbf{\textbf{relation}}ship is \textbf{\textbf{love}}, which is the foundation of any \textbf{\textbf{family}}. Without \textbf{\textbf{love}}, our \textbf{\textbf{family}} \textbf{\textbf{relation}}s can easily break down.

Ephesians 5:25 teaches us that \textbf{\textbf{husband}}s should \textbf{\textbf{love}} their wives as Christ \textbf{\textbf{love}}d the church, even laying down his life for it. It's a selfless kind of \textbf{\textbf{love}} that puts the needs of the other above one's own. Love between \textbf{\textbf{husband}} and \textbf{\textbf{wife}} is a precious gift that should never be taken for granted.

Respect is also a crucial element for a healthy \textbf{\textbf{family}} \textbf{\textbf{relation}}ship. In 1 Peter 3:7, \textbf{\textbf{husband}}s are instructed to show their wives \textbf{\textbf{respect}}, as they are the weaker partner. A \textbf{\textbf{husband}} who honors and \textbf{\textbf{respect}}s his \textbf{\textbf{wife}} is most likely to create an environment of trust and \textbf{\textbf{love}}. Similarly, wives are called to \textbf{\textbf{respect}} their \textbf{\textbf{husband}}s as the head of the \textbf{\textbf{family}}, as outlined in Ephesians 5:33.

Both \textbf{\textbf{husband}}s and wives need to understand the needs and \textbf{\textbf{expectation}}s of each other in a \textbf{\textbf{family}} \textbf{\textbf{relation}}ship. Communication is the key in meeting these \textbf{\textbf{expectation}}s. We ought to be slow to anger, quick to forgive and willing to listen. Instead of trying to \textbf{\textbf{control}} or coercing, we should strive to create an atmosphere of \textbf{\textbf{love}} and mutual \textbf{\textbf{respect}}.

Conviction is another vital component of a healthy \textbf{\textbf{relation}}ship. Sometimes, we may find it difficult to admit when we are wrong, but we should not let our pride stand in the way of reconciliation. The apostle Paul urged his disciples to speak the truth in \textbf{\textbf{love}}, as recorded in Ephesians 4:15. We should be open to correction and willing to confess our mistakes. It is only then that we can grow and mature in our \textbf{\textbf{relation}}ships.

Expectations are another area that can create conflicts in \textbf{\textbf{family}} \textbf{\textbf{relation}}s. Unspoken or unrealistic \textbf{\textbf{expectation}}s can cause misunderstandings or disappointments. In 1 Corinthians 13:4-8, Paul wrote a beautiful description of \textbf{\textbf{love}} that's patient, kind, not boastful, nor proud and doesn't keep a record of wrongs. If we have this kind of \textbf{\textbf{love}} in our hearts, we'll be able to navigate our \textbf{\textbf{expectation}}s in a mature and healthy way.

In conclusion, building strong \textbf{\textbf{family}} \textbf{\textbf{relation}}ships through \textbf{\textbf{love}} and \textbf{\textbf{respect}} is a challenging but deeply rewarding task. Let's strive to follow the example of Christ by loving one another sacrificially and showing \textbf{\textbf{respect}} to each other. Through open communication, mutual understanding, and \textbf{\textbf{convict}}ed hearts, we can create a harmonious home where \textbf{\textbf{love}} and \textbf{\textbf{respect}} reign.

Bible Verses Used: Ephesians 5:25, 1 Peter 3:7, Ephesians 5:33, Ephesians 4:15, 1 Corinthians 13:4-8
Bible Version: NIV

\end{document}